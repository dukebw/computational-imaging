\documentclass{article}

\title{Reaction Report to \em{Noise Flow: Noise Modeling with Conditional Normalizing Flows}}


\begin{document}

\maketitle

\section{Contribution}

A main contribution of Noise Flow is the proposal to use normalizing flows to
model camera sensor noise distributions.
This contribution stems from the recognition that real sensor noise
distributions have complex distributions.
The complexity of the real-world noise distribution derives from the
combination of multiple sources of noise with operations in the camera
pipeline.
For example, noise may be introduced by the discrete nature of photons (shot
noise), electronic noise, or gained noise.
Taken independently these sources of noise could be modeled by simple
distributions.
For instance, shot noise can be modeled by a Poisson process.
Taken together and combined with camera pipeline operations at different
stages, however, the overall distribution of noise becomes difficult to model.
This downstream noise is difficult to model even by a mixture of simple distributions.
Therefore Noise Flow contributes a method using normalizing flows to model
these complex noise distributions.
By using normalizing flows, the proposed method produces a density for the
noise, which is useful for both noise generation and likelihood computation.


\section{Extension}

Noise Flow proposes a generalization of simple noise models such
as the camera noise level function (NLF) and additive Gaussian noise.
It would be interesting to extend Noise Flow with the aim to improve
interpretability.
Knowing the distributions of each noise source would allow better diagnosis of
noise in camera pipelines.
To separate the final noise distribution into a mixture of noise from different
sources, we could collect a new dataset.
This new dataset could include readings at different stages of the camera
pipeline, before each noise source.
We could extend Noise Flow to have interpretable modules with known
distributions, such as a Poisson process for shot noise.
By learning to infer the parameters of the known distributions with our
augmented dataset, we could separate out the interpretable sources of noise
from the final noise distribution.
The augmented dataset would also lend a rigorous test for whether end-to-end
noise learning, as done by Noise Flow, can accurately attribute noise (e.g., to
Gaussian and Poisson components).


\end{document}
